% Configure URLs and hyperlinks
\hypersetup{
    colorlinks,
    linkcolor={red!50!black},
    citecolor={blue!50!black},
    urlcolor={blue!80!black}
}

% Create a pleasant code block style
\lstdefinestyle{paperStyle1}{
    backgroundcolor=\color{backcolour},
    commentstyle=\color{codegreen},
    keywordstyle=\color{magenta},
    numberstyle=\tiny\color{codegray},
    stringstyle=\color{codepurple},
    basicstyle=\ttfamily\footnotesize,
    breakatwhitespace=false,
    breaklines=true,
    captionpos=b,
    keepspaces=true,
    numbers=left,
    numbersep=5pt,
    showspaces=false,
    showstringspaces=false,
    showtabs=false,
    tabsize=2
}

% ===============================================================================
% From
% https://tex.stackexchange.com/questions/94902/how-can-i-use-line-numbers-with-leading-zeros
% Pads codeblock line numbers with zeros and aligns them correctly

\errorcontextlines=\maxdimen

\newcommand*{\boxeddecimalnum}[1]{\makebox[3em][r]{\decimalnum{#1}}}
\usepackage{refcount}
\usepackage{fmtcount}
\usepackage{intcalc}

\def\getnumdigitsaux #1{%
    \ifx#1\quark
        \expandafter\relax
    \else
        +1\expandafter\getnumdigitsaux
    \fi
}

\def\quark{\quark}
\newcommand\getnumdigits[1]{%
    \the\numexpr\getnumdigitsaux #1\quark
}

\newcounter{lstuniquenumber}
\makeatletter
\lst@AddToHook{Init}{\stepcounter{lstuniquenumber}\edef\lastlineincurrentlisting{\intcalcDec{\getrefnumber{lastlineinlisting\thelstuniquenumber}}}\padzeroes[\expandafter\getnumdigits\expandafter{\lastlineincurrentlisting}]} % at the start of each listing, count the listing, and try to restore the line number saved in the previous run
\lst@AddToHook{DeInit}{\label{lastlineinlisting\thelstuniquenumber}} % at the end of each listing, save one past the last line number
\makeatother

% ==============================================================================

\definecolor{myRed}{rgb}{0.96, 0.35, 0.40}
\definecolor{myGreen}{rgb}{0.65, 0.89, 0.18}
\definecolor{myBlue}{rgb}{0.25, 0.65, 0.99}
\definecolor{myPurple}{rgb}{0.78, 0.35, 0.91}
\definecolor{myGray}{rgb}{0.5, 0.5, 0.5}

\lstdefinestyle{lstStyle0}{
    autogobble,
    columns=fullflexible,
    showspaces=false,
    showtabs=false,
    breaklines=true,
    showstringspaces=false,
    breakatwhitespace=true,
    escapeinside={(*@}{@*)},
    identifierstyle=\color{myPurple},
    commentstyle=\color{myGreen},
    keywordstyle=\color{myBlue},
    stringstyle=\color{myRed},
    numberstyle=\color{myGray}\ttfamily\boxeddecimalnum,
    morekeywords={psrotate},
    basicstyle=\ttfamily\footnotesize,
    frame=1,
    numbers=left,
    framesep=22pt,
    tabsize=4,
    captionpos=b,
    xleftmargin=5.5ex
}

% Set style 0 as the default
\lstset{style=lstStyle0}

% Define a new tcolorbox style with line numbers
\tcbuselibrary{listingsutf8}
\tcbuselibrary{listings}

% Column separation default
\setlength{\columnsep}{1cm}

\tikzexternalize[prefix=tikz_figures/]
