\begin{figure*}
    \centering
    \begin{tikzpicture}
        \begin{axis}[
                colormap/PuOr,
            ]
            \addplot3+ [
                mesh,
                scatter,
                samples=10,
                domain=0:1,
            ] {x*(1-x)*y*(1-y)};
        \end{axis}
    \end{tikzpicture}
\end{figure*}

\begin{figure*}
    \centering
    \begin{tikzpicture}
        \begin{axis}[
                title={$x \exp(-x^2-y^2)$},
                domain=-2:2,
                colormap/viridis,
            ]
            \addplot3 [
                contour lua={
                        contour dir=y,
                        labels=false,
                        number=15,
                    },
                thick,
            ] {exp(-x^2-y^2)*x};
        \end{axis}
    \end{tikzpicture}
\end{figure*}

\begin{figure*}
    \centering
    \tdplotsetmaincoords{60}{130}
    \begin{tikzpicture}[tdplot_main_coords]
        % The function that is rotated
        \tikzmath{function f(\x) {return 1.5 - 0.325*sin(\x r);};}
        \pgfmathsetmacro{\dominio}{2.0}
        \pgfmathsetmacro{\xi}{-\dominio}
        \pgfmathsetmacro{\step}{(\dominio-\xi)/70.0}
        \pgfmathsetmacro{\xs}{\xi+\step}
        \pgfmathsetmacro{\max}{3}
        % Circumferences (behind the coordiante axis)
        \foreach \x in {\xi,\xs,...,\dominio}{
        \pgfmathsetmacro{\radio}{f(\x)}	% radius of the circumference of the solid of revolution
        \draw[cyan,very thick,opacity=0.35] plot[domain=0.5*pi:2.0*pi,smooth,variable=\t] ({\radio*cos(\t r)},{\x},{\radio*sin(\t r});
            }
        % Part of the solid of revolution behind the coordinate axis
        \foreach \angulo in {358,356,...,90}{
                \draw[cyan,very thick,rotate around y=\angulo,opacity=0.35] plot[domain=-\dominio:\dominio,smooth,variable=\t] ({0},{\t},{f(\t)});
            }
        % Graph of the function rotated about the $y$ axis
        \draw[red,ultra thick] plot[domain=-\dominio:\dominio,smooth,variable=\t] ({0},{\t},{f(\t)}) node [above right] {$z = f(y)$};
        % Coordinate axis
        \draw[thick,->] (0,0,0) -- (0,\max,0) node [right] {$y$};
        \draw[thick,->] (0,0,0) -- (\max,0,0) node [left] {$x$};
        \draw[thick,->] (0,0,0) -- (0,0,\max) node [above] {$z$};
        % Circumferences (in front of the coordiante axis)
        \foreach \x in {\xi,\xs,...,\dominio}{
        \pgfmathsetmacro{\radio}{f(\x)}	% Radio del círculo al inicio del sólido de revolución
        \draw[cyan,very thick,opacity=0.35] plot[domain=0.0:0.5*pi,smooth,variable=\t] ({\radio*cos(\t r)},{\x},{\radio*sin(\t r});
            }
        % The solid of revolution (in front of the coordinate axis)
        \foreach \angulo in {0,2,...,89}{
                \draw[cyan,very thick,rotate around y=\angulo,opacity=0.35] plot[domain=-\dominio:\dominio,smooth,variable=\t] ({0},{\t},{f(\t)});
            }
    \end{tikzpicture}
\end{figure*}
